\section{Optimal Control of Pitch/Travel with Feedback (LQ)}\label{sec:prob3}
TODO: Se template


% \section{Notater under dagen}
% Vi satt Q = diag[1 0 0 0], R = 2, og justerte Q_LQR.

% * Q_LQR justerer hvor mye vi vil straffe avvik mellom målt tilstands- og pådrags-trajektor og den beregnede optimale trajektoren.

% * R_LQR justerer hvor hardt vi vil straffe avvik mellom pådrag brukt og optimalt
% pådrag???
% TODO: Sjekk dette!

% Hvordan lage plots
% ------------------
% For hver kjøring med parametere beskrevet nedenfor, så ligger en
% tilsvarende measurements_q_xxxx.mat i measurements mappa. load
% denne i Matlab, plot optimal trajectory og plot målingene.

% Følg prosedyre i niceFigure.m for å plotte figurene i god kvalitet.

% Q_LQR = diag[1 1 1 1], R = diag[1]
% ----------------------------------
% Sammenligningsreferanse.

% Q_LQR = diag[20 1 1 1], R = diag[1]
% -----------------------------------
% Følger travel-trajektoren tettere, gav bra respons.

% Q_LQR = diag[50 1 1 1], R = diag[1]
% -----------------------------------
% Følger travel-trajektoren enda tettere, bedre respons.

% Q_LQR = diag[50 5 0 0], R = diag[1]
% -----------------------------------
% Ble ustabilt når vi tulla med pitch manuelt på slutten, siden
% vi ikke straffer bruk av pitch rate.

% Q_LQR = diag[50 1 20 1], R = diag[1]
% -----------------------------------
% Kommentar?

% Q_LQR = diag[1 1 10 1], R = diag[1]
% ----------------------------------
% Den optimale pitch-referanse trajektoren er basert på en dårlig modell.
% Så vi bør egentlig ikke følge den altfor tett. Følging av travel-trajektoren
% gir bedre resultat og kompenserer for modellfeil.

% Q_LQR = diag[1 1 40 1], R = diag[1]
% ----------------------------------
% Verre respons, siden vi følger en dårlig pådrags-trajektor.
